% defaults in emacs/modules/lang/
\documentclass[fontsize=10pt,paper=a4,paper=landscape,twoside=false,parskip=half,
headings=small,numbers=withenddot,usegeometry=true,english]{scrartcl}
\usepackage{mhoCheatSheet}
\usepackage{mhominted}
\usemintedstyle{tango}
\setminted{bgcolor=SnowII}
\usepackage{pagecolor}
\pagecolor{SnowIII}
%\nopagecolor %remove page color
\newfontfamily{\symbolfont}{Noto Sans Symbols 2}
\newcommand{\kcmdl}{\cBlue{L\symbolfont \Uchar"2318 }}
\newcommand{\meta}{\textbf{\cBlue{M}}}
\newcommand{\kcmdr}{\cPurple{R\symbolfont \Uchar"2318 }}
\newcommand{\ctrl}{\textbf{\cPurple{C}}}
\newcommand{\ktab}{\textbf{\cGreen{TAB}}}
\newcommand{\kcmd}{\textbf{\cRed{:\,}}}
\newcommand{\kshell}{\textbf{\cRed{:}\cBlue{!\,}}}
\newcommand{\kspc}{\cRed{SPC} \,}
\newcommand{\kleft}{\cSteelI{\symbolfont \Uchar"25C0} \,}
\newcommand{\kdown}{\cSteelI{\symbolfont \Uchar"25BC} \,}
\newcommand{\kup}{\cSteelI{\symbolfont \Uchar"25B2} \,}
\newcommand{\kright}{\cSteelI{\symbolfont \Uchar"25B6} \,}
\newcommand{\kenter}{\cSteelI{\symbolfont \Uchar"23CE} \,}
\newcommand{\kopt}{\cPink{\symbolfont \Uchar"2325}}
\newcommand{\kshift}{\cTeal{\symbolfont \Uchar"21E7}}
\newcommand{\kctrl}{\cOrange{\textasciicircum}}
\newcommand{\super}{\textbf{\cOrange{S}}}
\graphicspath{{\string~/Dropbox/assets/}}
\begin{document}

\begin{multicols}{3}

  \parbox{\columnwidth}{
    \begin{center}
      \includegraphics[width=0.2\columnwidth]{images/doom-emacs}
      \cPurple{\Huge Doom Emacs} \\[2pt]
      \cBlue{\large Unofficial Reference Sheet \today} \\[2pt]
    \end{center}

  }
  \section{Setup}
  \sectionbox{
    \subsection{Key Assignments: MacOS}
    The following code was added to \texttt{config.el}
    and intended for MacOS using a Swedish keyboard:
    \begin{minted}{lisp}
      (setq mac-option-modifier nil
      mac-command-modifier 'meta
      mac-right-command-modifier 'control
      mac-control-modifier 'super
      mac-function-modifier 'hyper)
    \end{minted}
    \begin{center}
      \fbox{\begin{minipage}{0.5cm}\hfill\kctrl \\ \tiny{control}\end{minipage}}\,\fbox{\begin{minipage}{0.5cm}\hfill\kopt \\ \tiny{option}\end{minipage}}\,\fbox{\begin{minipage}{0.7cm}\hfill\kcmdl \\ \tiny{command}\end{minipage}}\,\fbox{\begin{minipage}{3cm}\begin{center}\kspc \\ \tiny{space}\end{center}\end{minipage}}\,\fbox{\begin{minipage}{0.7cm}\kcmdr \\ \tiny{command}\end{minipage}} \,\fbox{\begin{minipage}{0.5cm}\kopt \\ \tiny{option}\end{minipage}}
    \end{center}
    \begin{center}
      \fbox{\begin{minipage}{0.5cm}\hfill\super \\ \tiny{super}\end{minipage}}\,\fbox{\begin{minipage}{0.5cm}\hfill\kopt \\ \tiny{option}\end{minipage}}\,\fbox{\begin{minipage}{0.7cm}\hfill\meta \\ \tiny{meta}\end{minipage}}\,\fbox{\begin{minipage}{3cm}\begin{center}\kspc \\ \tiny{space}\end{center}\end{minipage}}\,\fbox{\begin{minipage}{0.7cm}\ctrl \\ \tiny{control}\end{minipage}} \,\fbox{\begin{minipage}{0.5cm}\kopt \\ \tiny{option}\end{minipage}}
    \end{center}
  }
    \vspace{0.1cm}
  \begin{tblr}{
      row{odd} = {bg=SnowII},
      row{even} = {bg=SnowI},
      width=0.99\columnwidth,
      colspec={X[l] r},
    }
      \kspc & \cRed{Space} \\
      \kcmd & enter command mode \\
      \kcmdl & left command \( = \) \cBlue{Meta} \( = \) \meta \\
      \kcmdr & right command \( = \) \cPurple{Control} \( = \) \ctrl \\
      \kopt & left and right option keys \\
      \kshift & shift keys \\
      \kenter & enter \\
      \kctrl & control \( = \) \cOrange{Super} \( = \) \super \\
    \end{tblr}

  \section{Help}
  \begin{tblr}{
      row{odd} = {bg=SnowII},
      row{even} = {bg=SnowI},
      width=0.99\columnwidth,
      colspec={X[l] r},
    }
    \kspc h c & given keychord \( \rightarrow \) get command \\
    \kspc h f & describe functions \\
    \kspc h k & given keychord \( \rightarrow \) get command docs \\
    \kspc h w & given command \( \rightarrow \) get keychord \\
    \kspc h v & variable full docs \\
    \kspc h V & custom variable full docs \\
  \end{tblr}
  \section{Configuration Files}
  \sectionbox{
    \subsection{Configuration Files Location}
    Your configuration files: \cGreen{config.el} , \cGreen{init.el}
    , \cGreen{packages.el} are located in the directory: \mintinline{shell-session}{~/.config/doom/}
    \begin{itemize}
      \item \kspc f P \hfill browse private config files
      \item \kspc f p \hfill find file in private config files
    \end{itemize}
    \subsection{Changing Your Configuration Files}
    After making changes to any of your configuration file,
    \begin{enumerate}
      \item \kshell \mintinline{shell-session}{doom sync} \hfill run doom sync
      \item \kspc h r r \hfill reload doom emacs
    \end{enumerate}
  }

    \subsection{Reloading Configuration Files}
  \begin{tblr}{
      row{odd} = {bg=SnowII},
      row{even} = {bg=SnowI},
      width=0.99\columnwidth,
      colspec={X[l] r},
    }
       \kspc h r r & reload doom \\
       \kspc h r f & reload fonts \\
       \kspc h r t & reload theme \\
       \kspc h r p & reload packages \\
       \kspc h r e & reload environment \\
    \end{tblr}

  \sectionbox{
    \subsection{Add Doom to your Path}
    Add the following to your \cGreen{.zshrc} or \cGreen{.zprofile}
    in your home direcotry: \mintinline{shell-session}{~/}
    \begin{minted}{bash}
      # Add doom emacs to the Path
      export PATH=$HOME/.config/emacs/bin:$PATH
    \end{minted}
    After editing your \cGreen{.zshrc} or \cGreen{.zprofile} file, you
    need to reload it (source it) from your home directory in the terminal.
    \begin{minted}{shell-session}
      $ cd SPC ENTER
      $ source .zshrc
    \end{minted}
  }
  \section{Commands}
    \subsection{Run Emacs Commands}
  \begin{tblr}{
      row{odd} = {bg=SnowII},
      row{even} = {bg=SnowI},
      width=0.99\columnwidth,
      colspec={X[l] r},
    }
      \meta--x & run emacs commands -> the meta x \\
      \kspc : & run emacs commands -> the meta x \\
    \end{tblr}

    \subsection{Run Commands}
      \begin{tblr}{
      row{odd} = {bg=SnowII},
      row{even} = {bg=SnowI},
      width=0.99\columnwidth,
      colspec={X[l] r},
    }
      \kshell \mintinline{shell-session}{cmd-name [o-opt] <input>} & run shell command \\
      \kcmd \mintinline{shell-session}{yourVimCommand} & Vim command-line mode \\
    \end{tblr}

  \sectionbox{
    \subsection{Command Chords}
    \begin{center}\,\\
      \fbox{\cRed{OPERATOR}} \quad \fbox{\cBlue{COUNT}} \quad \fbox{\cGreen{MOTION}}
    \end{center}
    Example:
    \begin{itemize}
      \item \cRed{d} \cBlue{4} \cGreen{j} \hfill \cRed{delete} \cBlue{4} lines \cGreen{down}
    \end{itemize}
  }
  \section{Quit, Restart \& Save}
    \subsection{Quit}
  \begin{tblr}{
      row{odd} = {bg=SnowII},
      row{even} = {bg=SnowI},
      width=0.99\columnwidth,
      colspec={X[l] r},
    }
       \kspc q q & quit emacs \\
       \kspc q Q & quit emacs without saving \\
       \kcmd wq & write and quit \\
       \kcmd x & write and quit \\
       \kcmd q & quit -> throws error if unsaved changes \\
       \kcmd q! & discard unsaved changes and quit \\
       \kcmd wq & write (save) and quit \\
       Z Z & write and quit \\
       Z Q & discard unsaved changes and quit \\
    \end{tblr}
    \subsection{Restart}
  \begin{tblr}{
      row{odd} = {bg=SnowII},
      row{even} = {bg=SnowI},
      width=0.99\columnwidth,
      colspec={X[l] r},
    }
       \kspc q r & restart emacs and restore \\
       \kspc q R & restart emacs \\
    \end{tblr}
    \subsection{Save}
  \begin{tblr}{
      row{odd} = {bg=SnowII},
      row{even} = {bg=SnowI},
      width=0.99\columnwidth,
      colspec={X[l] r},
    }
     \kspc f s & save buffer \\
       \kspc b s & save buffer \\
       \kspc b S & save all open buffers \\
       \kcmd w & write (save) \\
    \end{tblr}
  \section{Cursor Motions}
  \sectionbox{
  Prefix cursor commands with a COUNT number, \#, to repeat the command
  \#-times.}
    \subsection{Basic Motions}
     \begin{tblr}{
      row{odd} = {bg=SnowII},
      row{even} = {bg=SnowI},
      width=0.99\columnwidth,
      colspec={X[l] r},
    }
       h or \kleft & move cursor left 1 character \\
       l or \kright & move cursor right 1 character \\
       j or \kdown & move cursor down 1 character \\
       k or \kup & move cursor up 1 character \\
    \end{tblr}

    \subsection{Word Motions}
     \begin{tblr}{
      row{odd} = {bg=SnowII},
      row{even} = {bg=SnowI},
      width=0.99\columnwidth,
      colspec={X[l] r},
    }
       b & jump backwards start word \\
       B & jump backwards (punctuated words) start word \\
       e & jump forwards end word \\
     g e & jump backwards end previous word \\
     g E & jump backwards (punctuated words) previous word \\
       E & jump forwards (punctuated words) end word \\
       w & jump forwards start next word \\
       W & jump forwards (punctuated words) start next word \\
     \kshift \_ & jump to first non-blank character of line \\
    \end{tblr}

    \subsection{Buffer Motions}
  \begin{tblr}{
      row{odd} = {bg=SnowII},
      row{even} = {bg=SnowI},
      width=0.99\columnwidth,
      colspec={X[l] r},
    }
     G & jump to start of buffer \\
     g g & jump to end of buffer \\
    \end{tblr}

    \subsection{Find Motions}
  \begin{tblr}{
      row{odd} = {bg=SnowII},
      row{even} = {bg=SnowI},
      width=0.99\columnwidth,
      colspec={X[l] r},
    }
     f ? & jump to next occurance of character ? \\
     t ? & jump to before next occurance of character ? \\
     F ? & jump to previous occurance of character ? \\
     T ? & jump to after previous occurance of character ? \\
     ; & repeat f, t, F, T movement forwards \\
     , & repeat f, t, F, T movement backwards \\
       \kshift 5 & jump to matching delimiter in pair (), \{\}, [] \\
    \end{tblr}

    \subsection{Paragraph Motions}
  \begin{tblr}{
      row{odd} = {bg=SnowII},
      row{even} = {bg=SnowI},
      width=0.99\columnwidth,
      colspec={X[l] r},
    }
     \kshift \kopt 8 & jump to previous paragraph, function or block \\
     \kshift \kopt 9 & jump to next paragraph, function or block \\
  \end{tblr}

    \subsection{Screen Motions}
  \begin{tblr}{
      row{odd} = {bg=SnowII},
      row{even} = {bg=SnowI},
      width=0.99\columnwidth,
      colspec={X[l] r},
    }
     H & move top of screen \\
     L & move bottom of screen \\
     M & move middle of screen \\
     \ctrl--b & move screen up one page \\
     \ctrl--e & move screen down one line \\
     \ctrl--f & move screen down one page \\
     \ctrl--d & move cursor \& screen down 1/2 page \\
     \ctrl--u & move cursor \& screen up 1/2 page \\
     \ctrl--y & move screen up one line \\
     z z & center cursor on screen \\
     z t & position cursor at top of screen \\
     z b & position cursor at bottom of screen \\
    \end{tblr}

    \subsection{Function Motions}
  \begin{tblr}{
      row{odd} = {bg=SnowII},
      row{even} = {bg=SnowI},
      width=0.99\columnwidth,
      colspec={X[l] r},
    }
     g d & move to local declaration \\
     g D & move to global declaration \\
    \end{tblr}

\section{Insert Mode}
  \subsection{Exit}
   \begin{tblr}{
      row{odd} = {bg=SnowII},
      row{even} = {bg=SnowI},
      width=0.99\columnwidth,
      colspec={X[l] r},
    }
     \cRed{ESC} & exit insert mode \\
     \ctrl--g & exit insert mode \\
  \end{tblr}

  \subsection{Cursors}
  \begin{tblr}{
      row{odd} = {bg=SnowII},
      row{even} = {bg=SnowI},
      width=0.99\columnwidth,
      colspec={X[l] r},
    }
     a & append after the cursor \\
     i & insert before the cursor \\
  \end{tblr}

  \subsection{Macros}
    \begin{tblr}{
      row{odd} = {bg=SnowII},
      row{even} = {bg=SnowI},
      width=0.99\columnwidth,
      colspec={X[l] r},
    }
     q & stop recording a macro \\
     q a & record macro \\
     @a & run macro a \\
     @@ & rerun last macro \\
  \end{tblr}

  \subsection{Lines}
    \begin{tblr}{
      row{odd} = {bg=SnowII},
      row{even} = {bg=SnowI},
      width=0.99\columnwidth,
      colspec={X[l] r},
    }
     o  append new line below current line \\
     A  insert at end of line \\
     O  append new line above current line \\
     I  insert at beginning of line \\
     \ctrl--t  indent line \\
     \ctrl--d  de-indent line \\
     \ctrl--j  add line break current position \\
  \end{tblr}

\section{Buffers}

  \subsection{Viewing Buffers}
    \begin{tblr}{
      row{odd} = {bg=SnowII},
      row{even} = {bg=SnowI},
      width=0.99\columnwidth,
      colspec={X[l] r},
    }
     \kspc b b & view workspace buffers \\
     \kspc , & view workspace buffers \\
     \kspc b B & view all buffers \\
     \kspc b i & view all ibuffer \\
     \kspc b I & view workspace ibuffer \\
  \end{tblr}

  \subsection{Navigating Buffers}
    \begin{tblr}{
      row{odd} = {bg=SnowII},
      row{even} = {bg=SnowI},
      width=0.99\columnwidth,
      colspec={X[l] r},
    }
     \kspc b n & next buffer \\
     \kspc b l & previous buffer \\
  \end{tblr}

  \subsection{Creating and Deleting Buffers}
    \begin{tblr}{
      row{odd} = {bg=SnowII},
      row{even} = {bg=SnowI},
      width=0.99\columnwidth,
      colspec={X[l] r},
    }
     \kspc b c & clone current buffer \\
     \kspc b N & new empty buffer \\
     \kspc b d & kill (delete) buffer \\
     \kspc b k & kill (delete) buffer \\
     \kspc b K & kill (delete) ALL buffers \\
     \kspc b O & kill (delete) other buffers \\
  \end{tblr}

\section{Search \& Replace}
  \subsection{Search and Replace}
  \begin{tblr}{
      row{odd} = {bg=SnowII},
      row{even} = {bg=SnowI},
      width=0.99\columnwidth,
      colspec={X[l] r},
    }
    \* & select current word forwards \\
    \# & select current word backwards \\
    \cRed{/}pattern & search forwards for pattern \\
    \cRed{?}pattern & search backwards for pattern \\
    \kcmd noh & remove highlighting of search \\
    n & search next same direction \\
    N & search next opposite direction \\
    \kcmd \%s/old/new/g & replace all old with new \\
    \kcmd \%s/old/new/gc & replace all old with new confirm \\
  \end{tblr}

\section{Windows}
  \subsection{Creating Windows}
       \begin{tblr}{
      row{odd} = {bg=SnowII},
      row{even} = {bg=SnowI},
      width=0.99\columnwidth,
      colspec={X[l] r},
    }
    \kspc w n & new window \\
    \kspc w s & split window \\
    \kspc w S & split window and navigate to window \\
  \end{tblr}

  \subsection{Navigating Windows}
    \begin{tblr}{
      row{odd} = {bg=SnowII},
      row{even} = {bg=SnowI},
      width=0.99\columnwidth,
      colspec={X[l] r},
    }
     \kspc w j & navigate to window below \\
     \kspc w h & navigate to window left \\
     \kspc w k & navigate to window above \\
     \kspc w l & navigate to window right \\
  \end{tblr}

  \subsection{Moving Windows}
     \begin{tblr}{
      row{odd} = {bg=SnowII},
      row{even} = {bg=SnowI},
      width=0.99\columnwidth,
      colspec={X[l] r},
    }
     \kspc w J & move window down \\
     \kspc w H & move window left \\
     \kspc w K & move window up \\
     \kspc w L & move window right \\
  \end{tblr}

  \subsection{Change Window Size}
     \begin{tblr}{
      row{odd} = {bg=SnowII},
      row{even} = {bg=SnowI},
      width=0.99\columnwidth,
      colspec={X[l] r},
    }
     \kspc w + & increase window height \\
     \kspc w - & decrease window height \\
     \kspc w < & increase window width \\
     \kspc w < & decrease window width \\
     \kspc w = & balance windows \\
  \end{tblr}

\columnbreak%
\section{Normal Mode}

  \subsection{Repeat, Undo and Replace}
     \begin{tblr}{
      row{odd} = {bg=SnowII},
      row{even} = {bg=SnowI},
      width=0.99\columnwidth,
      colspec={X[l] r},
    }
     . & repeat last command \\
     r & replace single character \\
     R & replace characters until ESC \\
     u & undo \\
     \ctrl--r & redo \\
  \end{tblr}

  \subsection{Copy and Paste}
     \begin{tblr}{
      row{odd} = {bg=SnowII},
      row{even} = {bg=SnowI},
      width=0.99\columnwidth,
      colspec={X[l] r},
    }
     p & put (paste) clipboard after cursor \\
     y & yank (copy) \\
     P & paste before cursor \\
     Y & yank to end of line \\
     y w & yank the characters from cursor to next word \\
     y y & yank line \\
     x p & transpose two characters \\
     y a w & yank word and space before or after \\
     y i w & yank word \\
  \end{tblr}

  \subsection{Delete}
     \begin{tblr}{
      row{odd} = {bg=SnowII},
      row{even} = {bg=SnowI},
      width=0.99\columnwidth,
      colspec={X[l] r},
    }
     x & delete character \\
     D & delete to end of line \\
     d d & delete line \\
     d w & delete word from cursor \\
     d i w & delete word \\
     \kcmd n,md & delete lines starting from n to m \\
     \kcmd .,\kopt 4 d & delete from current line to end file \\
     \kcmd .,1 d & delete from current line to start file \\
     \kcmd n,1 d & delete from line n to start file \\
  \end{tblr}

  \subsection{Change}
     \begin{tblr}{
      row{odd} = {bg=SnowII},
      row{even} = {bg=SnowI},
      width=0.99\columnwidth,
      colspec={X[l] r},
    }
     C & change to end of line \\
     c c & change line \\
     c w & change to end of word \\
     c i w & change word \\
  \end{tblr}

  \subsection{Transpose Characters and Line Joins}

     \begin{tblr}{
      row{odd} = {bg=SnowII},
      row{even} = {bg=SnowI},
      width=0.99\columnwidth,
      colspec={X[l] r},
    }
     J & join line below with line with space \\
     g J & join line below with line with no space \\
     x p & transpose two characters \\
  \end{tblr}

  % https://github.com/hlissner/evil-multiedit
  \subsection{Evil Multi-Edit}

     \begin{tblr}{
      row{odd} = {bg=SnowII},
      row{even} = {bg=SnowI},
      width=0.99\columnwidth,
      colspec={X[l] r},
    }
     D & clear the region \\
     C & clear to end of region and go into insert mode \\
     A & go into insert mode at end of region \\
     I & go into insert mode at start of region \\
     R & highlight all matches of selection in buffer \\
     P & replace highlighted region with the clipboard \\
     V & select the region \\
     \$ & go to end of region \\
     0 & go to start of region \\
     gg & go to first region \\
     G & go to last region \\
     \kenter & toggle selected region under cursor \\
     \meta--d & match next selected region \\
     \meta--D & match previous selected region \\
     \ctrl--n & move to next selected region \\
     \ctrl--N & move to previous selected region \\
     \meta--\ctrl--D & restore last groupe multi-edit regions \\
  \end{tblr}

  \subsection{Text Indentation}

     \begin{tblr}{
      row{odd} = {bg=SnowII},
      row{even} = {bg=SnowI},
      width=0.99\columnwidth,
      colspec={X[l] r},
    }
     > > & move indent right \\
     < < & move indent left (de-indent) \\
     > \% & indent () or \{\} block (cursor on bracket) \\
     < \% & de-indent () or \{\} block (cursor on bracket) \\
     > i b & indent inner block \\
     > i B & re-indent inner block \\
     > a t & indent block with < > tags \\
     = \% & re-indent () or \{\} block (cursor on bracket) \\
     g g = G & re-indent entire buffer \\
  \end{tblr}

  \subsection{Text Folding}
     \begin{tblr}{
      row{odd} = {bg=SnowII},
      row{even} = {bg=SnowI},
      width=0.99\columnwidth,
      colspec={X[l] r},
    }
     z a & toggle folds \\
     z o & open folds \\
     z c & close fold \\
     z C & outline hide subtree (org) \\
     z m & hide next fold level \\
     z M & close all folds \\
     z r & fold less \\
     z x & update folds \\
     z v & open fold this line \\
     z i & toggle inline images \\
     z A & toggle folds recursively \\
     z C & close folds recursively \\
     z M & close all \\
     z o & open fold \\
     z O & open fold recursively \\
     z r & open next fold \\
     z R & open all folds \\
  \end{tblr}



\section{Visual Mode}
  \subsection{Visual Mode}
  \sectionbox{
    Entering Visual Mode
    }
     \begin{tblr}{
      row{odd} = {bg=SnowII},
      row{even} = {bg=SnowI},
      width=0.99\columnwidth,
      colspec={X[l] r},
    }
     v & start visual mode, mark region  \\
     V & start visual-line mode \\
     \ctrl v & start visual-block mode \\
     A & insert mode at end of region \\
     I & insert mode at start of region \\
     o & toggle cursor to other end of marked region \\
     \cRed{ESC} & exit visual mode \\
  \end{tblr}

  \subsection{Mark Regions}
     \begin{tblr}{
      row{odd} = {bg=SnowII},
      row{even} = {bg=SnowI},
      width=0.99\columnwidth,
      colspec={X[l] r},
    }
     a w & mark a word \\
     a b & mark a () block\\
     a ( & mark a () block\\
     a B & mark a \{\} block\\
     a \textbraceleft & mark a \{\} block\\
     a t & mark a <> tags block\\
     i b & mark the inner of () block\\
     i B & mark the inner of \{\} block\\
     i p & mark the inner of paragraph\\
     i t & mark the inner of <> tags block\\
  \end{tblr}

  \subsection{Surround Mode}

     \begin{tblr}{
      row{odd} = {bg=SnowII},
      row{even} = {bg=SnowI},
      width=0.99\columnwidth,
      colspec={X[l] r},
    }
     S & surround a region of text \\
     S <tagName \kenter & surround region with given tag name \\
     S ( \kenter & surround region () \\
  \end{tblr}

  \subsection{Actions on Regions}

     \begin{tblr}{
      row{odd} = {bg=SnowII},
      row{even} = {bg=SnowI},
      width=0.99\columnwidth,
      colspec={X[l] r},
    }
     d & delete marked text \\
     y & yank (copy) marked text \\
     u & change to lowercase \\
     U & change to uppercase \\
     \kopt .. & switch case (key beside return key) \\
  \end{tblr}

\section{Python}
  \subsection{Running Python}

     \begin{tblr}{
      row{odd} = {bg=SnowII},
      row{even} = {bg=SnowI},
      width=0.99\columnwidth,
      colspec={X[l] r},
    }
     \meta--x run-python & start a python REPL \\
     \ctrl--c \meta--o & clear python REPL \\
  \end{tblr}

\section{Customizations}

  \subsection{Themes}

     \begin{tblr}{
      row{odd} = {bg=SnowII},
      row{even} = {bg=SnowI},
      width=0.99\columnwidth,
      colspec={X[l] r},
    }
     \kspc h t & choose (consult) theme \\
     \kspc h r t & reload theme \\
  \end{tblr}

\columnbreak%
\section{Dired - File Management}

  \subsection{Starting Dired}

     \begin{tblr}{
      row{odd} = {bg=SnowII},
      row{even} = {bg=SnowI},
      width=0.99\columnwidth,
      colspec={X[l] r},
    }
     \kspc f d & open directory in dired (prompts you) \\
     \kspc o - & open current directory in dired \\
     \kspc p . & open project root in dired \\
     q & kill all dired buffers \\
  \end{tblr}

  \subsection{Directory Navigation}

     \begin{tblr}{
      row{odd} = {bg=SnowII},
      row{even} = {bg=SnowI},
      width=0.99\columnwidth,
      colspec={X[l] r},
    }
     \kenter & Visit the file/directory new buffer \\
     - & go up one directory \\
     ( & toggle file info detail \\
     ) & dired-git-info-mode \\
     = & diff files \\
     a & visit file/directory same buffer \\
     g r & refresh in the dired directory \\
     g o & visit the file in the other directory \\
  \end{tblr}

  \subsection{Marking Files \& Directories}

     \begin{tblr}{
      row{odd} = {bg=SnowII},
      row{even} = {bg=SnowI},
      width=0.99\columnwidth,
      colspec={X[l] r},
    }
     d & mark file(s) for deletion \\
     m & mark files \\
     q & kill all dired buffers \\
     u & unmark a file/directory \\
     U & unmark all marked file/directory \\
     \% m & mark files by regexp \\
     ! & run shell command on marked files \\
  \end{tblr}

  \subsection{Creating \& Delete}

     \begin{tblr}{
      row{odd} = {bg=SnowII},
      row{even} = {bg=SnowI},
      width=0.99\columnwidth,
      colspec={X[l] r},
    }
     \kspc . & create new file \\
     \kspc f f & create new file \\
     + & create directory \\
     D & delete the file at point \\
     x & delete marked files \\
     Z & zip (compress) a file/directory \\
  \end{tblr}

  \subsection{Copy \& Rename}

     \begin{tblr}{
      row{odd} = {bg=SnowII},
      row{even} = {bg=SnowI},
      width=0.99\columnwidth,
      colspec={X[l] r},
    }
     C & copy the file at point \\
     q & kill all dired buffers \\
     R & rename the file at point \\
     = & diff files \\
  \end{tblr}

  \subsection{Bulk Renaming files/directories}
  \sectionbox{Steps renaming files/directories in bulk}

     \begin{tblr}{
      row{odd} = {bg=SnowII},
      row{even} = {bg=SnowI},
      width=0.99\columnwidth,
      colspec={X[l] r},
    }
     \textbf{Step 1:} \ctrl--c, \ctrl--e & open writable grep buffer \\
     \textbf{Step 2:} & rename files \& directories in the buffer. \\
     \textbf{Step 3:} \ctrl--c, \ctrl--c & commit changes \\
     \\
     \ctrl--c, \ctrl--k & abort changes at any time \\
  \end{tblr}

\columnbreak%

\section{Org-Mode}
  \subsection{Headings}

     \begin{tblr}{
      row{odd} = {bg=SnowII},
      row{even} = {bg=SnowI},
      width=0.99\columnwidth,
      colspec={X[l] r},
    }
     \kshift--\kenter & create headline of same type below \\
     \kshift--\ctrl--\kenter & create headline of same type above \\
     \meta--h & move heading left (promote) \\
     \meta--l & move heading right (demote) \\
     \meta--k & move heading up \\
     \meta--j & move heading down \\
     \kshift--\meta--h & move heading and children left \\
     \kshift--\meta--l & move heading and children right \\
     \kshift--\meta--k & move heading and children up \\
     \kshift--\meta--j & move heading and children down \\
  \end{tblr}

  \subsection{Links}

     \begin{tblr}{
      row{odd} = {bg=SnowII},
      row{even} = {bg=SnowI},
      width=0.99\columnwidth,
      colspec={X[l] r},
    }
     \kspc m l c & clipboard link \\
    % \kspc m l g & macos link \\
     \kspc m l l & create link \\
     \kspc m l d & delete link \\
     \kspc m l s & store link \\
     \kspc m l S & insert last stored link \\
     \kspc m l t & toggle link display \\
  \end{tblr}

  \subsection{Babel Source Code Blocks}

     \begin{tblr}{
      row{odd} = {bg=SnowII},
      row{even} = {bg=SnowI},
      width=0.99\columnwidth,
      colspec={X[l] r},
    }
     src \ktab & create a babel source block \\
     \ctrl--c \ctrl--c & run babel block \\
     \kspc m k & remove babel output \\
     \kspc m K & remove all babel output \\
  \end{tblr}
  \sectionbox{
\begin{minted}{elisp}
#+begin_src web
;; your html code goes here
#+end_src
\end{minted}
}

  \subsection{Org-Agenda}

     \begin{tblr}{
      row{odd} = {bg=SnowII},
      row{even} = {bg=SnowI},
      width=0.99\columnwidth,
      colspec={X[l] r},
    }
     \kspc m d & agenda deadline \\
     \kspc m s & agenda schedule \\
  \end{tblr}

  \subsection{Org-Agenda Clock}

     \begin{tblr}{
      row{odd} = {bg=SnowII},
      row{even} = {bg=SnowI},
      width=0.99\columnwidth,
      colspec={X[l] r},
    }
     \kspc m c c & clock cancel \\
     \kspc m c g & goto clock \\
     \kspc m c i & clock in \\
     \kspc m c o & clock out \\
     \kspc m c r & clock report mode \\
     \kspc m c s & show clocking issues \\
  \end{tblr}

  \subsection{Calendar Minibuffer}

     \begin{tblr}{
      row{odd} = {bg=SnowII},
      row{even} = {bg=SnowI},
      width=0.99\columnwidth,
      colspec={X[l] r},
    }
     \ctrl--h & backward 1 day \\
     \ctrl--l & forward 1 day \\
     \ctrl--k & backward 1 week \\
     \ctrl--j & forward 1 week \\
     \kshift--\ctrl--h & backward 1 month \\
     \kshift--\ctrl--l & forward 1 month \\
     \kshift--\ctrl--k & backward 1 year \\
     \kshift--\ctrl--j & forward 1 year \\
  \end{tblr}

  \subsection{Org Tree/Subtree}

     \begin{tblr}{
      row{odd} = {bg=SnowII},
      row{even} = {bg=SnowI},
      width=0.99\columnwidth,
      colspec={X[l] r},
    }
     \kspc m s b & tree to indirect buffer \\
     \kspc m s c & clone subtree with time shift \\
     \kspc m s d & cut subtree \\
     \kspc m s h & promote subtree \\
     \kspc m s j & move subtree down \\
     \kspc m s k & move subtree up \\
     \kspc m s l & demote subtree \\
     \kspc m s n & narrow to subtree \\
     \kspc m s N & widen \\
     \kspc m s r & org refile \\
     \kspc m s s & sparse tree \\
     \kspc m s S & org sort \\
     \kspc m s A & archive subtree \\
  \end{tblr}

\end{multicols}
\end{document}
